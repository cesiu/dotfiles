% Because I'm too lazy to manage my used packages properly.


% Document Setup:
\usepackage[english]{babel}
\usepackage{fontspec}
\usepackage{lastpage}
\usepackage{epigraph}
% Figures:
\usepackage{caption}
\usepackage{framed}
\usepackage{mdframed}
% Lists and Tables:
\usepackage{mdwlist}
\usepackage{booktabs}
\usepackage{multicol}
\usepackage{multirow}
\usepackage{array}
\usepackage{arydshln}
% Algorithms and Code:
\usepackage{algorithm}
\usepackage[noend]{algpseudocode}
\usepackage{listings}
\usepackage{cprotect}
% Math:
\usepackage{mathtools}
\usepackage{mathrsfs}
\usepackage{amsmath}
\usepackage{amssymb}
\usepackage{amsthm}
\usepackage{lualatex-math}
% Drawings and Images:
\usepackage{tikz}
\usepackage{graphicx}
\usepackage{xcolor}
\usepackage{phaistos}
\usepackage{textgreek}
% Misc. Text Formatting:
\usepackage{hyperref}
\usepackage{url}
\usepackage{cancel}
\usepackage{setspace}
\usepackage{microtype}


% Fixes older bits of Classic Thesis:
\renewcommand{\sc}{\scshape}
\setlength{\epigraphwidth}{0.5\linewidth}
\newcommand{\tinyskip}{\vspace{1.5pt plus 0.5pt minus 0.5pt}}

% Fixed-width tabular columns:
\newcolumntype{C}[1]{>{\centering\arraybackslash}m{#1}}
\newcolumntype{L}[1]{>{\raggedright\arraybackslash}m{#1}}
\newcolumntype{R}[1]{>{\raggedleft\arraybackslash}m{#1}}

% More Beamer-friendly theorem blocks:
\theoremstyle{definition}\newtheorem{theorem*}{Theorem}
\theoremstyle{definition}\newtheorem{corollary*}{Corollary}
\theoremstyle{definition}\newtheorem{lemma*}{Lemma}
\theoremstyle{example}\newtheorem{excont}{Example \textit{(cont.)}}

% I like thin arrows in my propositions:
\DeclareMathOperator{\contradiction}{\rightarrow\leftarrow}
\let\implies\relax
\DeclareMathOperator{\implies}{\rightarrow}
\let\iff\relax
\DeclareMathOperator{\iff}{\leftrightarrow}
\newcommand{\fn}[2]{#1\hspace{-0.1em}\left(#2\right)}

% Extra algorithm statements:
\renewcommand{\algorithmiccomment}[1]{\hspace{2em}// #1}
\algnewcommand\Input{\item[\textbf{Input:}]}
\algnewcommand\Output{\item[\textbf{Output:}]}
\algnewcommand{\Let}[1]{\State{\textbf{let} #1}}
\renewcommand{\algorithmicreturn}[1]{\State{\textbf{return}#1}}
% Algorithm statements for plain-English-style pseudocode:
\algnewcommand{\BigLet}[1]{\State{Let #1}}
\algnewcommand{\BigRet}[1]{\State{Return #1}}
\algloopdefx{BigFor}[1]{For #1}
\algloopdefx{BigIf}[1]{If #1}
%\algcloopdefx[<new loop>]{<old block>}{<continue>}
%   [<continueparamcount>][<default value>]{<continue text>}

% Temporary variables:
\newdimen{\tempx}
\newdimen{\tempy}

% TikZ libraries:
\usetikzlibrary{
    positioning,
    calc,
    fit,
    shapes,
    shapes.multipart,
    shapes.misc,
    intersections,
    arrows,
    arrows.meta,
    decorations.pathreplacing,
    decorations.pathmorphing,
    automata,
    chains
}

% Listings settings:
\lstset{
    numbers=left,
    numberstyle=\scriptsize\ttfamily,
    basicstyle=\normalsize\ttfamily,
    basewidth=0.55em,
    upquote=true,
    frame=L,
    escapeinside=``
}


% Deprecated, but some of the 225 materials from Winter '18 still use them.
%\lstdefinestyle{skippable}{
%    numberblanklines=false
%}
%\lstdefinestyle{output}{
%    moredelim = [is][\itshape]{|}{|}
%}
%\lstdefinestyle{soln}{
%    moredelim = [is][\color{red}\itshape]{|}{|}
%}
%\let\origthelstnumber\thelstnumber
%\makeatletter
%\newcommand*\SuspendNumber{
%    \lst@AddToHook{OnNewLine}{
%    \let\thelstnumber\relax
%        \advance\c@lstnumber-\@ne\relax
%    }
%}
%
%\newcommand*\ResumeNumber[1]{
%    \setcounter{lstnumber}{\numexpr#1-1\relax}
%    \lst@AddToHook{OnNewLine}{
%        \let\thelstnumber\origthelstnumber
%        \refstepcounter{lstnumber}
%    }
%}
%\makeatother
